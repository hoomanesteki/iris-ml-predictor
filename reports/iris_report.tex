% Options for packages loaded elsewhere
% Options for packages loaded elsewhere
\PassOptionsToPackage{unicode}{hyperref}
\PassOptionsToPackage{hyphens}{url}
\PassOptionsToPackage{dvipsnames,svgnames,x11names}{xcolor}
%
\documentclass[
  letterpaper,
  DIV=11,
  numbers=noendperiod]{scrartcl}
\usepackage{xcolor}
\usepackage{amsmath,amssymb}
\setcounter{secnumdepth}{-\maxdimen} % remove section numbering
\usepackage{iftex}
\ifPDFTeX
  \usepackage[T1]{fontenc}
  \usepackage[utf8]{inputenc}
  \usepackage{textcomp} % provide euro and other symbols
\else % if luatex or xetex
  \usepackage{unicode-math} % this also loads fontspec
  \defaultfontfeatures{Scale=MatchLowercase}
  \defaultfontfeatures[\rmfamily]{Ligatures=TeX,Scale=1}
\fi
\usepackage{lmodern}
\ifPDFTeX\else
  % xetex/luatex font selection
\fi
% Use upquote if available, for straight quotes in verbatim environments
\IfFileExists{upquote.sty}{\usepackage{upquote}}{}
\IfFileExists{microtype.sty}{% use microtype if available
  \usepackage[]{microtype}
  \UseMicrotypeSet[protrusion]{basicmath} % disable protrusion for tt fonts
}{}
\makeatletter
\@ifundefined{KOMAClassName}{% if non-KOMA class
  \IfFileExists{parskip.sty}{%
    \usepackage{parskip}
  }{% else
    \setlength{\parindent}{0pt}
    \setlength{\parskip}{6pt plus 2pt minus 1pt}}
}{% if KOMA class
  \KOMAoptions{parskip=half}}
\makeatother
% Make \paragraph and \subparagraph free-standing
\makeatletter
\ifx\paragraph\undefined\else
  \let\oldparagraph\paragraph
  \renewcommand{\paragraph}{
    \@ifstar
      \xxxParagraphStar
      \xxxParagraphNoStar
  }
  \newcommand{\xxxParagraphStar}[1]{\oldparagraph*{#1}\mbox{}}
  \newcommand{\xxxParagraphNoStar}[1]{\oldparagraph{#1}\mbox{}}
\fi
\ifx\subparagraph\undefined\else
  \let\oldsubparagraph\subparagraph
  \renewcommand{\subparagraph}{
    \@ifstar
      \xxxSubParagraphStar
      \xxxSubParagraphNoStar
  }
  \newcommand{\xxxSubParagraphStar}[1]{\oldsubparagraph*{#1}\mbox{}}
  \newcommand{\xxxSubParagraphNoStar}[1]{\oldsubparagraph{#1}\mbox{}}
\fi
\makeatother


\usepackage{longtable,booktabs,array}
\usepackage{calc} % for calculating minipage widths
% Correct order of tables after \paragraph or \subparagraph
\usepackage{etoolbox}
\makeatletter
\patchcmd\longtable{\par}{\if@noskipsec\mbox{}\fi\par}{}{}
\makeatother
% Allow footnotes in longtable head/foot
\IfFileExists{footnotehyper.sty}{\usepackage{footnotehyper}}{\usepackage{footnote}}
\makesavenoteenv{longtable}
\usepackage{graphicx}
\makeatletter
\newsavebox\pandoc@box
\newcommand*\pandocbounded[1]{% scales image to fit in text height/width
  \sbox\pandoc@box{#1}%
  \Gscale@div\@tempa{\textheight}{\dimexpr\ht\pandoc@box+\dp\pandoc@box\relax}%
  \Gscale@div\@tempb{\linewidth}{\wd\pandoc@box}%
  \ifdim\@tempb\p@<\@tempa\p@\let\@tempa\@tempb\fi% select the smaller of both
  \ifdim\@tempa\p@<\p@\scalebox{\@tempa}{\usebox\pandoc@box}%
  \else\usebox{\pandoc@box}%
  \fi%
}
% Set default figure placement to htbp
\def\fps@figure{htbp}
\makeatother


% definitions for citeproc citations
\NewDocumentCommand\citeproctext{}{}
\NewDocumentCommand\citeproc{mm}{%
  \begingroup\def\citeproctext{#2}\cite{#1}\endgroup}
\makeatletter
 % allow citations to break across lines
 \let\@cite@ofmt\@firstofone
 % avoid brackets around text for \cite:
 \def\@biblabel#1{}
 \def\@cite#1#2{{#1\if@tempswa , #2\fi}}
\makeatother
\newlength{\cslhangindent}
\setlength{\cslhangindent}{1.5em}
\newlength{\csllabelwidth}
\setlength{\csllabelwidth}{3em}
\newenvironment{CSLReferences}[2] % #1 hanging-indent, #2 entry-spacing
 {\begin{list}{}{%
  \setlength{\itemindent}{0pt}
  \setlength{\leftmargin}{0pt}
  \setlength{\parsep}{0pt}
  % turn on hanging indent if param 1 is 1
  \ifodd #1
   \setlength{\leftmargin}{\cslhangindent}
   \setlength{\itemindent}{-1\cslhangindent}
  \fi
  % set entry spacing
  \setlength{\itemsep}{#2\baselineskip}}}
 {\end{list}}
\usepackage{calc}
\newcommand{\CSLBlock}[1]{\hfill\break\parbox[t]{\linewidth}{\strut\ignorespaces#1\strut}}
\newcommand{\CSLLeftMargin}[1]{\parbox[t]{\csllabelwidth}{\strut#1\strut}}
\newcommand{\CSLRightInline}[1]{\parbox[t]{\linewidth - \csllabelwidth}{\strut#1\strut}}
\newcommand{\CSLIndent}[1]{\hspace{\cslhangindent}#1}



\setlength{\emergencystretch}{3em} % prevent overfull lines

\providecommand{\tightlist}{%
  \setlength{\itemsep}{0pt}\setlength{\parskip}{0pt}}



 


\KOMAoption{captions}{tableheading}
\makeatletter
\@ifpackageloaded{caption}{}{\usepackage{caption}}
\AtBeginDocument{%
\ifdefined\contentsname
  \renewcommand*\contentsname{Table of contents}
\else
  \newcommand\contentsname{Table of contents}
\fi
\ifdefined\listfigurename
  \renewcommand*\listfigurename{List of Figures}
\else
  \newcommand\listfigurename{List of Figures}
\fi
\ifdefined\listtablename
  \renewcommand*\listtablename{List of Tables}
\else
  \newcommand\listtablename{List of Tables}
\fi
\ifdefined\figurename
  \renewcommand*\figurename{Figure}
\else
  \newcommand\figurename{Figure}
\fi
\ifdefined\tablename
  \renewcommand*\tablename{Table}
\else
  \newcommand\tablename{Table}
\fi
}
\@ifpackageloaded{float}{}{\usepackage{float}}
\floatstyle{ruled}
\@ifundefined{c@chapter}{\newfloat{codelisting}{h}{lop}}{\newfloat{codelisting}{h}{lop}[chapter]}
\floatname{codelisting}{Listing}
\newcommand*\listoflistings{\listof{codelisting}{List of Listings}}
\makeatother
\makeatletter
\makeatother
\makeatletter
\@ifpackageloaded{caption}{}{\usepackage{caption}}
\@ifpackageloaded{subcaption}{}{\usepackage{subcaption}}
\makeatother
\usepackage{bookmark}
\IfFileExists{xurl.sty}{\usepackage{xurl}}{} % add URL line breaks if available
\urlstyle{same}
\hypersetup{
  pdftitle={Flower Species Classification Based on Iris Dataset},
  pdfauthor={Suryash Chakravarty, Hooman Esteki \& Bright Arafat Bello},
  colorlinks=true,
  linkcolor={blue},
  filecolor={Maroon},
  citecolor={Blue},
  urlcolor={Blue},
  pdfcreator={LaTeX via pandoc}}


\title{Flower Species Classification Based on Iris Dataset}
\author{Suryash Chakravarty, Hooman Esteki \& Bright Arafat Bello}
\date{2025-12-06}
\begin{document}
\maketitle


GitHub URL: https://github.com/hoomanesteki/iris-ml-predictor

\subsection{Summary}\label{summary}

The present endeavor constructs a model that classifies iris flower
species through the classic Iris dataset Fisher (1936). It was a
four-dimensional feature set, which consisted of sepal length, sepal
width, petal length, and petal width that was employed to get a Decision
Tree Classifier. The model's performance was determined using a separate
test set, and accuracy levels of 93.33\% were observed, which is quite a
strong one.

The Iris dataset is often utilized for teaching machine learning because
of its uncomplicated nature and distinct class structure. Nevertheless,
the limitations of the small dataset containing 150 samples and the
overlap of features between \emph{versicolor} and \emph{virginica} make
it harder to generalize. Our model, regardless of these restrictions,
still manages to exhibit high classification performance and serve as a
robust baseline for multilabel prediction tasks.

NB: Some of the code for our analysis was adapted from courses at the
Masters Of Data Science program at UBC, particularly; 1. DSCI 571:
Supervised Learning I; UBC Master of Data Science Program (2025b) 2.
DSCI 522: Data Science Workflows; UBC Master of Data Science Program
(2025a)

\subsection{Introduction}\label{introduction}

The 150 samples in the Iris dataset are determined by four numerical
characteristics which together give the dimensions of the iris flowers.
The target variable consists of three species: \emph{Iris setosa},
\emph{Iris versicolor}, and \emph{Iris virginica}. In the main, the
analysis has the aim of identifying if the machine learning model---in
this case, a Decision Tree Classifier---can make correct predictions of
species identity relying only on these measurements.

The machine learning process outlined in this report is an entire cycle
consisting of data exploration, cleaning, and transformation, modeling
and finally, evaluation. Access to the full code and scripts that were
used for the analysis is provided through the GitHub repository:
https://github.com/hoomanesteki/iris-ml-predictor.

\subsection{Methods}\label{methods}

\subsubsection{Data Source and
Preprocessing}\label{data-source-and-preprocessing}

The collection of data was obtained from the UCI Machine Learning
Repository Fisher (1936). There are no missing values in the dataset,
and 50 samples represent each of the three classes. The class labels
were converted into numbers (0 = setosa, 1 = versicolor, 2 = virginica).
A train-test split Pedregosa et al. (2011), was done to maintain the
separation of training and testing data. This will help evaluate the
model's performance on previously unseen data.

\subsubsection{Exploratory Data
Analysis}\label{exploratory-data-analysis}

Seaborn Waskom (2021) was used to create exploratory visualizations to
know the distributions and separability of features:

\begin{itemize}
\tightlist
\item
  Pairplot to see class clustering Figure~\ref{fig-pplot}
\end{itemize}

\begin{figure}

\centering{

\includegraphics[width=0.7\linewidth,height=\textheight,keepaspectratio]{../results/figures/pairplot.png}

}

\caption{\label{fig-pplot}Pairplot of Features vs Target}

\end{figure}%

\begin{itemize}
\tightlist
\item
  Correlation heatmap to find correlations among quantitative features
  Figure~\ref{fig-corr}
\end{itemize}

\begin{figure}

\centering{

\includegraphics[width=0.7\linewidth,height=\textheight,keepaspectratio]{../results/figures/corr.png}

}

\caption{\label{fig-corr}Heatmap of Correlation (1.0 = High +ve
Correlation, 0 = No Correlation)}

\end{figure}%

\begin{itemize}
\tightlist
\item
  Distribution plot indicating differences in petal length among species
  Figure~\ref{fig-hist}
\end{itemize}

\begin{figure}

\centering{

\includegraphics[width=0.7\linewidth,height=\textheight,keepaspectratio]{../results/figures/histplot.png}

}

\caption{\label{fig-hist}Histogram of Classes vs Petal Width}

\end{figure}%

The visualizations indicate that \emph{setosa} is completely isolated
from the remaining two species whereas \emph{versicolor} and
\emph{virginica} have slightly overlapping areas.

\subsubsection{Model Building}\label{model-building}

A DummyClassifier (actually a classifier with no intelligence at all)
was employed as a reference point for the performance comparison (nearly
33\% correct predictions for the three equally balanced classes).
Subsequently, a Decision Tree Classifier was fitted to capture the
nonlinear patterns through the four input features.

\begin{longtable}[]{@{}lllll@{}}

\caption{\label{tbl-metrics}Decision Tree Classifier Performance
Metrics}

\tabularnewline

\caption{}\label{T_e036c}\tabularnewline
\toprule\noalign{}
~ & accuracy & precision\_weighted & recall\_weighted & f1\_weighted \\
\midrule\noalign{}
\endfirsthead
\toprule\noalign{}
~ & accuracy & precision\_weighted & recall\_weighted & f1\_weighted \\
\midrule\noalign{}
\endhead
\bottomrule\noalign{}
\endlastfoot
0 & 0.933333 & 0.950000 & 0.933333 & 0.934762 \\

\end{longtable}

\subsection{Results \& Discussion}\label{results-discussion}

The Decision Tree classifier produced a test accuracy of 93.33\%, which
is a considerable improvement over the baseline dummy model. This is a
strong indication that the model was able to recognize and utilize the
underlying patterns in the data.

\begin{figure}

\centering{

\includegraphics[width=0.7\linewidth,height=\textheight,keepaspectratio]{../results/figures/confusion_matrix.png}

}

\caption{\label{fig-conf}Confusion matrix}

\end{figure}%

The confusion matrix (Figure~\ref{fig-conf}) highlights:

\begin{itemize}
\tightlist
\item
  The classification of \emph{setosa} is perfect.
\item
  There are some misclassifications between \emph{versicolor} and
  \emph{virginica}, which is in line with the feature distributions that
  overlap.
\end{itemize}

\subsubsection{Interpretation}\label{interpretation}

Setosa is distinguished without doubt by its distinct petal features.
The mentioned mix up between \emph{versicolor} and \emph{virginica}
indicates:

\begin{itemize}
\tightlist
\item
  Feature intersection hampers the linear or rule based separation
\item
  A more complex or regularized decision tree might be beneficial
\item
  More sophisticated models (such as Random Forest, SVM) might have
  better performance
\end{itemize}

\subsubsection{Future Work}\label{future-work}

In order to elevate the model's power and generalization:

\begin{itemize}
\tightlist
\item
  Tune the hyperparameters (maximum depth, minimum samples split)
\item
  Try out other classifiers and evaluate them
\item
  Apply k-fold cross-validation to obtain a stability of results
\item
  Look into the importance of features and concentrates on misclassified
  samples
\item
  Consider the use of ensemble models as a means of improving robustness
\end{itemize}

\subsection*{References}\label{references}
\addcontentsline{toc}{subsection}{References}

\phantomsection\label{refs}
\begin{CSLReferences}{1}{0}
\bibitem[\citeproctext]{ref-iris_53}
Fisher, R. A. 1936. {``Iris.''} UCI Machine Learning Repository.

\bibitem[\citeproctext]{ref-scikit_learn}
Pedregosa, F., G. Varoquaux, A. Gramfort, V. Michel, B. Thirion, O.
Grisel, M. Blondel, et al. 2011. {``Scikit-Learn: Machine Learning in
{P}ython.''} \emph{Journal of Machine Learning Research} 12: 2825--30.

\bibitem[\citeproctext]{ref-dsci522_milestone}
UBC Master of Data Science Program. 2025a. {``DSCI 522: Data Science
Workflows --- Milestone Instructions.''}
\url{https://pages.github.ubc.ca/mds-2024-25/DSCI_522_dsci-workflows_students/}.

\bibitem[\citeproctext]{ref-dsci571_materials}
---------. 2025b. {``DSCI 571: Supervised Learning i --- Course
Materials.''}
\url{https://pages.github.ubc.ca/mds-2025-26/DSCI_571_sup-learn-1_students/learning_objectives.html}.

\bibitem[\citeproctext]{ref-Waskom2021}
Waskom, Michael L. 2021. {``Seaborn: Statistical Data Visualization.''}
\emph{Journal of Open Source Software} 6 (60): 3021.
\url{https://doi.org/10.21105/joss.03021}.

\end{CSLReferences}




\end{document}
